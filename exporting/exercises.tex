
    \documentclass{lib/unichristusdoc}

    \usepackage{amsmath}

    \usepackage[utf8]{inputenc}

    \usepackage[portuguese]{babel}

    \def\course{Fundamentos de Matemática da Computação III}

    \def\prof{John Smith}

    \def\semester{2025.1}

    \def\codeCourse{IMD1234}

    \def\registration{}

    \def\student{}

    \def\graduate{Tecnologia da Informação}

    \def\theme{Lista de exercícios 1}
\begin{document}

    \makeheader

    \fbox{

    \parbox{\textwidth}{

    \begin{minipage}{\textwidth}

    \makeinstructions

    {

    \begin{instlist}

    \item Preencha o cabeçalho da folha pergunta com seus dados.

    \item Todas as folhas respostas devem conter o nome a a matrícula do aluno.

    \item O preenchimento das respostas deve ser feito utilizando caneta (preta ou azul).

    \end{instlist}

    }

    \end{minipage}

    }

    }
\vspace{1cm}\problem Verifique se $  ( ( P \lor Q ) \lor ( Q \land P ) ) \land ( P \land P )  $ pode ser concluído partindo das premisas: 

\subproblem $  ( P \lor P ) \land ( Q \lor P )  $

\subproblem $  P \lor ( ( P \land ( P \land Q ) ) \land P )  $

\subproblem $  ( ( Q \lor Q ) \lor ( Q \land Q ) ) \land ( Q \lor P )  $

\vspace{1cm}\problem Verifique se $  ( ( Q \land Q ) \land P ) \land ( Q \land Q )  $ pode ser concluído partindo das premisas: 

\subproblem $  ( ( P \lor P ) \lor P ) \lor ( Q \land P )  $

\subproblem $  ( ( P \land ( Q \lor Q ) ) \land P ) \lor ( Q \lor Q )  $

\subproblem $  ( ( Q \land P ) \lor P ) \land ( P \land P )  $

\vspace{1cm}\problem Verifique se $  Q \lor ( P \lor ( Q \land ( Q \lor P ) ) )  $ pode ser concluído partindo das premisas: 

\subproblem $  ( ( Q \land P ) \lor ( P \land P ) ) \land ( P \lor Q )  $

\subproblem $  ( Q \land P ) \land P  $

\subproblem $  P \land ( P \land ( ( Q \lor P ) \land ( Q \land P ) ) )  $

\vspace{1cm}\problem Verifique se $  ( P \lor P ) \land Q  $ pode ser concluído partindo das premisas: 

\subproblem $  ( ( ( P \lor Q ) \land ( P \land Q ) ) \lor P ) \lor P  $

\subproblem $  ( ( ( Q \land Q ) \lor Q ) \land P ) \land Q  $

\subproblem $  ( P \lor P ) \lor ( P \land Q )  $

\vspace{1cm}\problem Verifique se $  ( ( P \land Q ) \land ( Q \lor P ) ) \land P  $ pode ser concluído partindo das premisas: 

\subproblem $  Q \land ( P \land ( P \lor Q ) )  $

\subproblem $  ( ( ( ( P \lor P ) \lor Q ) \land Q ) \lor P ) \lor P  $

\subproblem $  P \land ( ( P \lor P ) \lor Q )  $

\end{document}